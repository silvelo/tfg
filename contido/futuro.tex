\chapter{Trabajo Futuro}
\label{chap:futuro}

En este capítulo se comentarán algunos de los posibles desarrollos que se podrían llevar a cabo en un futuro. 

\begin{itemize}
    \item \textbf{Refinar el proceso de extracción y selección de características}: Partiendo de la implementación actual, se propone explorar nuevas características en busca de una optimización de los resultados.
    
    \item \textbf{Integración con aplicaciones de terceros}: Parte del producto final del proyecto es una aplicación propia que recoge los eventos, abstraer este concepto en forma de librería permitiría integrarla en aplicaciones de terceros.
    
    \item \textbf{Actualización de los modelos}: Este tipo de sistemas suelen estar entrenados con un conjunto de datos iniciales que es persistente en el tiempo, pero los seres humanos pueden cambiar su comportamiento a lo largo de su vida. Por lo tanto, un aprendizaje que evolucionase con el usuario permitiría obtener una mayor tolerancia a este problema.
    
    \item \textbf{Explorar nuevas técnicas}: Las técnicas utilizadas a lo largo del proyecto son algunas de las múltiples posibilidades existentes dentro del marco de la \textit{IA}, por lo que realizar pruebas con otros métodos para comparar resultados e intentar mejorar el rendimiento, sería una buena forma de darle continuidad al proyecto.
    

\end{itemize}

De todas las opciones de trabajo futuro comentadas, conseguir que el algoritmo pueda adaptarse a lo largo del tiempo al usuario, es la más interesante. Pues se conseguiría que el algoritmo pueda seguir reconociendo al usuario, en cualquier momento de su vida y ante cualquier circunstancia externa (enfermedad, estrés~\dots)
