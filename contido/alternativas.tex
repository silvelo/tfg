\section{Estudio de alternativas}
\label{Estudio de alternativas}
Debido al interés de este tipo de sistemas de autenticación, se pueden encontrar múltiples estudios relacionados. Existen un número importante de estudios/proyectos sobre la autenticación basada en el uso de un dispositivo móvil y los sensores que este contiene. Pero la mayoría de estos estudios/proyecto no consiguen obtener un producto final~\cite{EHATISHAMULHAQ201824}~\cite{JAIN2019604}~\cite{DAMOPOULOS2019138}~\cite{SMITHCREASEY2018147}~\cite{YANG20199}. 

Las alternativas que tienen o muestran un producto final usable se mencionan en las siguientes secciones del capitulo.

\subsection{SealSign}

En el \textit{Mobile World Congress} del año 2015 se dio a conocer \textit{SealSign}~\cite{sealsign} por \textbf{Eleven Paths}, una aplicación que permite la firma manuscrita en los dispositivos móviles, mediante el análisis de parámetros biométricos para validar al usuario. Estos parámetros, según los desarrolladores, son la velocidad y la presión en los ejes X e Y, de los cuales obtienen la huella biométrica de esa firma. Este sistema cuenta con la posibilidad de realizar firmas masivas de documentos y, además, permite una integración con cualquier otro lenguaje de manera sencilla.


\subsection{Biosig-id}
\textit{Biosig-id}~\cite{biosig-id} es una herramienta de seguridad que no solo valida la contraseña del usuario, sino que además analiza el comportamiento de escritura tanto con ratón como por pantalla táctil. Esta herramienta cuenta con aplicación de demostración~\footnote{\url{https://biosig-id.com/go-verify-yourself/}} en la que retan al usuario a falsificar una firma, dando una recompensa al que consiga entrar.


\subsection{Touch me once and I know it's you }

En el año 2012 un grupo de la Universidad de Múnich, desarrolló una aplicación~\cite{de2012touch} que capturaba los movimientos del usuario cuando estos realizaban el patrón para desbloquear el móvil. La aplicación consistía en repetir ochenta veces seguidas el patrón de desbloqueo, para evitar la monotonía del proceso, se realizaban pausas cada veinte patrones, en las cuales el usuario debía de realizar otro tipo de acciones durante un periodo largo de tiempo.




