%%%%%%%%%%%%%%%%%%%%%%%%%%%%%%%%%%%%%%%%%%%%%%%%%%%%%%%%%%%%%%%%%%%%%%%%%%%%%%%%

\begin{abstract}\thispagestyle{empty}


Los usuarios de aplicaciones móviles con temática social, financiera o personal quieren que sus datos estén siempre protegidos sin necesidad de realizar procesos o configuraciones complicadas. Por ello, un buen sistema de seguridad es crucial para evitar que los datos sean modificados o robados. Los sistemas de autenticación son una de las principales barreras para evitar este tipo de situaciones.

Con este objetivo, se han implementado a lo largo de los años métodos de autenticación que permitan al sistema verificar si un usuario es quién dice ser y, por lo tanto, permitir al usuario realizar ciertas acciones dentro del sistema. Los métodos de autenticación biométricos son los que más destacan entre los existentes, estos pretenden identificar a un usuario basándose en sus propias características, tanto físicas (huella dactilar, retina\dots) como de comportamiento (escritura, voz\dots).

En este proyecto se plantea un estudio para comprobar la viabilidad de un sistema biométrico extrayendo los patrones de comportamiento de un usuario a la ahora de usar un dispositivo móvil, mediante el análisis de la iteración de este individuo con la pantalla táctil y los sensores de movimiento.

Para este propósito, se ha planteado un sistema de autenticación basado en el uso de perfiles personales construidos mediante técnicas de Inteligencia Artificial, a partir de eventos capturados por una aplicación implementada para este propósito.



  \vspace*{25pt}
  \begin{segundoresumo}
Users of mobile applications with social, financial or personal themes want their data be protected without a complicated process or configuration. Therefore, a good security system is crucial to prevent data from being modified, corrupted or stolen. Authentication systems are one of the main barriers to avoid this situation.

With this aim, authentication methods have been implemented over the years so that they allow the system to know if a user is who claims to be and, therefore, grant access permissions to legitimate users. The biometric authentication methods are the ones that stand out among the existing ones, these are intended to identify a user based on their own characteristics, both physical (fingerprint, retina\dots) and behavioral (writing, voice\dots).

This project proposes a study to check the viability of a biometric system by extracting behavioral patterns of the user when using a mobile device by analyzing the iteration of this individual with the touch screen and motion sensors.

For this purpose, we propose an authentication system based on the use of personal profiles constructed using Artificial Intelligence techniques, based on events captured by an application implemented for this purpose.

  \end{segundoresumo}
\vspace*{25pt}
\begin{multicols}{2}
\begin{description}
\item [\palabraschaveprincipal:] \mbox{} \\[-20pt]
\begin{itemize}
    \item Inteligencia Artificial
    \item Seguridad Informática
    \item Sistemas biométricos
    \item Sistema Autenticación
    \item Aplicación Híbrida
    \item NodeJS
    \item Red Neuronal Artificial
    
\end{itemize}

\end{description}
\begin{description}
\item [\palabraschavesecundaria:] \mbox{} \\[-20pt]
\begin{itemize}
    \item Artificial Intelligence
    \item Computer Security
    \item Biometric Systems
    \item Authentication System
    \item Hybrid Application
    \item NodeJS
    \item Artificial Neural Networks
\end{itemize}
\end{description}
\end{multicols}

\end{abstract}
%%%%%%%%%%%%%%%%%%%%%%%%%%%%%%%%%%%%%%%%%%%%%%%%%%%%%%%%%%%%%%%%%%%%%%%%%%%%%%%%
